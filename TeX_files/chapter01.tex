\chapter{Introduction}

Hand pose are one of the most common communication methods in daily human life. Interaction modalities of user interfaces play a dominant role in the relationship between people and computer technology. The way users interact with interfaces has undergone a significant transformation, with most of the population now using touch devices. In today's technological advancements, human-computer interfaces (HCI) are highly appealing, mainly because they aim to enhance human lifestyles. Alongside these developments, technologies that facilitate interaction with these devices are also required. Initially, interactions were conducted using a mouse and keyboard with computers. However, with the rise of ubiquitous computing, gestures \cite{9427388} have become more prevalent—for example, smartphone interactions often involve hand gestures. The human body provides a wide range of poses that can be used as computer input. Images captured by cameras exhibit a vast amount of variation. This occurs because images are spatial data where each pixel represents a color at a specific coordinate. For pose classification, many images are needed for training to account for variations in scale, position, and hand orientation within the images. The desired outcome is a feature invariant to scale, position, and orientation, ensuring accurate and robust classification.

\section{Importance of Hand Pose Recognition}

Hand pose recognition is pivotal in advancing human-computer interaction (HCI), enabling intuitive communication between humans and machines. It bridges the gap between natural human gestures \cite{9609102} and machine understanding, opening new avenues in virtual reality, robotics, and accessibility technologies. For instance, hand gesture recognition is essential for sign language translation, empowering people with hearing impairments to communicate seamlessly with others. Moreover, the ability to accurately recognize hand poses enhances precision in applications such as gaming, remote robotic control, and augmented reality, where fine motor movements dictate user experience. As technology continues to evolve, the importance of hand pose recognition grows, particularly in creating inclusive, adaptive, and user-friendly systems.

\section{Challenges in Hand Pose Recognition}

Despite its significance, hand pose recognition presents numerous challenges, primarily due to the variability and complexity of human hands. Hands have intricate structures with multiple degrees of freedom, leading to a vast range of poses that can be difficult to capture and interpret accurately. Variations in lighting, occlusions caused by overlapping fingers, and differences in hand shapes across individuals add further complexity. Real-time processing demands also pose a challenge, requiring high computational efficiency without sacrificing accuracy. Additionally, datasets used for training hand pose recognition models often need more diversity, limiting their generalizability in real-world scenarios. Addressing these challenges requires robust algorithms capable of handling variations and noise while maintaining computational efficiency.

\section{Hand Pose Recognition Implementation in Medical}
Hand pose recognition is revolutionizing in many sectors such as non-verbal communication \cite{ZHANG20242399}, patient care, and therapy methods especially in the medical field. Surgeons can utilize gesture-based systems to interact with medical imaging during procedures, eliminating the need for physical contact with equipment and ensuring sterility in the operating room. In rehabilitation, hand pose recognition monitors and guides patients recovering from injuries or surgeries involving fine motor skills. For example, it enables therapists to track real-time progress and customize exercises based on a patient's movements. Moreover, hand pose recognition aids in developing assistive devices for individuals with motor impairments, enhancing their ability to perform daily tasks independently. This technology's application in healthcare improves the precision of treatments and fosters patient engagement and empowerment.